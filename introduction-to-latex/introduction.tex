\documentclass{article}
\usepackage{pgf}
\usepackage{pgfpages}
\usepackage{hyperref}
\title{introduction}
\author{Emilian 'synnek' Zawrotny}
\date{6th December 2019}

\pgfpagesdeclarelayout{boxed}
{
  \edef\pgfpageoptionborder{0pt}
}
{
  \pgfpagesphysicalpageoptions
  {%
    logical pages=1,%
  }
  \pgfpageslogicalpageoptions{1}
  {
    border code=\pgfsetlinewidth{2pt}\pgfstroke,%
    border shrink=\pgfpageoptionborder,%
    resized width=.95\pgfphysicalwidth,%
    resized height=.95\pgfphysicalheight,%
    center=\pgfpoint{.5\pgfphysicalwidth}{.5\pgfphysicalheight}%
  }%
}

\pgfpagesuselayout{boxed}
\begin{document}
\begin{abstract}
\includegraphics[scale=0.1]{./GCI-new-logo.jpg}
This article is made for Google Code-In 2019 Challange for Fedora Project.
Google Code-In is a competition for students within age 13-17 years organized by Google.
During GCI students around the world works for Open-Source organizations contributing to their projects.
It is very good chance for students to learn more about Programming and UI/UX Design. The competition starts
every year and lasts 2 months. The prizes are: T-Shirts, Jackets, Online Certificates and best students wins
a trip to Google's office in San Francisco. If you' re a student passionated in IT, I really recommend you trying your best in GCI.

\end{abstract}
Hello. My name is Emilian, but on the internet I have been known as 'synnek'. I am 15 years old. My hobby is widely-described IT - especially coding and security, but also videomaking and VFX/Motion Design.
Currently I am participating in Google Code-In 2019 Challange for Fedora Organization. But in a past, I have made a few projects that I think can intrest you. You can look at them on my \href{https://github.com/synnek1337/}{GitHub}
\end{document}